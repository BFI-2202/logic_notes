\documentclass{article}

\usepackage[T2A]{fontenc}
\usepackage[utf8]{inputenc}
\usepackage[russian]{babel}

\usepackage{tabularx}
\usepackage{amsmath}
\usepackage{pgfplots}
\usepackage{geometry}
\usepackage{multicol}
\geometry{
    left=1cm,right=1cm,top=2cm,bottom=2cm
}
\newcommand*\diff{\mathop{}\!\mathrm{d}}

\newtheorem{definition}{Определение}
\newtheorem{theorem}{Теорема}

\DeclareMathOperator{\sign}{sign}

\usepackage{hyperref}
\hypersetup{
    colorlinks, citecolor=black, filecolor=black, linkcolor=black, urlcolor=black
}

\title{Математическая логика и теория алгоритмов}
\author{Лисид Лаконский}
\date{September 2023}

\begin{document}
\raggedright

\maketitle

\tableofcontents
\pagebreak

\section{Лекция — 22.09.2023}

\subsection{Исчисление высказываний}

\begin{definition}
    \textbf{Формальное исчисление} $I$ определено, если выполняются следующие условия:
    \begin{enumerate}
        \item Имеется некоторое множество $A(I)$ — алфавит исчислениия $I$. Его элементы называются символами. Конечные последовательности символов называются словами исчисления.Через $W(I)$ обозначим множество всех слов алфавита исчисления $I$.
        \item Пусть задано множество $E(I)$, которое является подмножеством $W(I)$, называемое множеством выражений исчисления $I$. Элементы множества $E(I)$ называются формулами (секвенциями).
        \item Выделено подмножество $A_x(I)$, которое является подможеством $E(I)$ выражений исчисления $I$, называемое аксиомами исчисления $I$.
        \item Имеется конечное множество $R(I)$ частичных операций $R_{1}, R_{2}, \dots, R_{n}$ на множестве $E(I)$, называемых правилами вывода исчислений $I$.
    \end{enumerate}
    Таким образом, любое исчисления определяется четырьмя подмножествами: $A(I)$, $E(I)$, $A_{x}(I)$, $R(I)$.
\end{definition}

Если набор выражения $(\Phi_{1}, \Phi_{2}, \dots, \Phi_{m}, \Phi)$ принадлежит правилу $R_{i}$, то выражение $\Phi_{1}, \dots, \Phi_{m}$ называются посылками, а выражение $\Phi$ называется непосредственным следствием выражения $\Phi_{1}, \dots, \Phi_{m}$ по правилу $R_{i}$ (заключением правила $R_{i}$) и записывается следующим образом: $\frac{\Phi_{1}, \dots, \Phi_{m}}{\Phi} i$ (индекс $i$ может опускаться, если понятно, о каком правиле идет речь)

\begin{definition}
\textbf{Выводом исчисления} $I$ называется последовательность выражений $\Phi_{1}, \Phi_{2}, \dots, \Phi_{n}$ такая, что для любого $1 \le i \le n$ выражение $\Phi_{i}$ есть либо аксиома исчисления $I$, либо непосредственное следствие каких-либо предыдущих выражений.
\end{definition}

\begin{definition}
Выражение $\Phi$ называется \textbf{теоремой исчисления} $I$, вводимой или доказываемой в $I$, если существует вывод $\Phi_{1}, \Phi_{2}, \dots, \Phi_{n}, \Phi$, который называется выводом выражения $\Phi$ или доказательством теоремы $\Phi$.
\end{definition}

Пусть $E(I)$ — множество программ $P_{f}$, производящих вычисление значений одноместных числовых функций $f$, а $A_{x}(I)$ — \textbf{множество простых программ}. Пусть $P_{f} P_{g} \subseteq E(I)$. $P_{f} \circ P_{g}$ обозначим программу, по начальным данным которой вычисляется функция $f$, и потом вывод которой используется в качестве начальных функции $g$. Правило вывода обозначается следующим образом: $\frac{P_{f} ; P_{g}}{P_{f} \circ P_{g}}$.

\textbf{В общем случае не существует алгоритм}, с помощью которого можно для произвольного выражений $\Phi$ формального исчисления $I$, за конечное число определить, является ли $\Phi$ выводимым в $I$, или нет.

Если такой алгоритм существует, то исчисление называется \textbf{разрешимым}, если же такой алгоритм не существует, то исчисление называется \textbf{неразрешимым}.

Исчисления называется \textbf{непротиворечивым}, если не все его выражения доказуемые.

Рассмотрим исчисления, лежащие в основе алгебры логики:

\begin{enumerate}
    \item \textbf{Исчисление высказываний Генценовского типа, предложенных Генценом}. В качестве выражений используются секвенции, построенные из формул алгебры логики. Эти исчисления будем обозначать как ИС.
    \item \textbf{Исчисление высказываний Гильбертовского типа, предложенных Гильбертом}. В нем выражениями являются непосредственно формулы алгебры логики. Эти исчисления будем обозначать как ИВ.
\end{enumerate}

Эти исчисления \textbf{являются эквивалентными} в том смысле, что \textbf{доказуемыми в них являются тождественно истинные формулы}.

\subsubsection{Исчисление высказываний Генценовского типа}

\textbf{Алфавит исчисления} $A(I)$ состоит из букв $A,B,Q,P,R,\dots$ и возможно других букв, и возможно с индексами, которые называются \textbf{пропозиционными элементами}. Определены следующие логические символы-связки: отрицание ($\lnot$), конъюнкция ($\land$), дизъюнкция ($\lor$), импликация ($\rightarrow$), следование ($\vdash$), вспомогательные символы $($ $)$ и $,$

\textbf{Множество формул} исчисления высказываний Генценовского типа определяется индуктивно:

\begin{enumerate}
    \item Все пропозиционные переменные являются формулами исчисления высказываний. Такие формулы называются \textbf{элементарными} (атомарными).
    \item Пусть $\phi$, $\psi$ есть некоторые формулы в рассматриваемом нами исчислении высказываний. Любые формулы, построенные из этих формул с помощью логических операций \textbf{также являются формулами данного исчисления высказываний}.
\end{enumerate}

\textbf{Выражение является формулой исчисления высказываний} тогда и только тогда, когда это может быть установлено с пунктов 1 и 2.

\textbf{Секвенциями} называются конечные выражения следующих двух видов, где $\phi_1, \dots, \phi_{n}, \psi$ — формулы исчисления высказываний:

\begin{enumerate}
    \item $\phi_1, \dots, \phi_{n} \vdash \psi$ (из истинности формул $\phi_1, \dots, \phi_{n}$ следует $\psi$)
    \item $\phi_1, \dots, \phi_{n} \vdash$ (система формул $\phi_1, \dots, \phi_{n}$ является противоречивой)
\end{enumerate}

Последовательность формул $\phi_1, \dots, \phi_{n}$ будем обозначать через $\Gamma$. В нашем случае получается:

\begin{enumerate}
    \item $\Gamma \vdash \psi$
    \item $\Gamma \vdash$
\end{enumerate}

При этом последовательность $\Gamma$ считается пустой при $n = 0$. Запись $\vdash \psi$ и $\vdash$ \textbf{также являются секвенциями}, первая из которых может читаться как «утверждение доказуемости формулы $\psi$.

Таким образом, наряду с формулами, символизирующими простые или сложные высказывания, секвенцией являются записи утверждений, в которых выделяются посылки и заключения.

\textbf{Множество аксиом} исчисления высказываний Генценовского типа \textbf{определяется следующей схемой секвенций}:

\begin{enumerate}
    \item $\phi \vdash \phi \ \forall \ \phi \in E(I)$.
\end{enumerate}

\end{document}