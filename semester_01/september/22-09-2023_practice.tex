\documentclass{article}

\usepackage[T2A]{fontenc}
\usepackage[utf8]{inputenc}
\usepackage[russian]{babel}

\usepackage{tabularx}
\usepackage{amsmath}
\usepackage{pgfplots}
\usepackage{geometry}
\usepackage{multicol}
\geometry{
    left=1cm,right=1cm,top=2cm,bottom=2cm
}
\newcommand*\diff{\mathop{}\!\mathrm{d}}

\newtheorem{definition}{Определение}
\newtheorem{theorem}{Теорема}

\DeclareMathOperator{\sign}{sign}

\usepackage{hyperref}
\hypersetup{
    colorlinks, citecolor=black, filecolor=black, linkcolor=black, urlcolor=black
}

\title{Математическая логика и теория алгоритмов}
\author{Лисид Лаконский}
\date{September 2023}

\begin{document}
\raggedright

\maketitle

\tableofcontents
\pagebreak

\section{Практическое занятие — 22.09.2023}

Рассмотрим функцию четырех переменных $f(x_1, x_2, x_3, x_4) = x_1 x_2 \lor x_3 x_4 \lor \overline{x_1} \overline{x_2} x_3 \lor x_1 \overline{x_3} \overline{x_4} \lor x_2 \overline{x_3} \overline{x_4}$.

Требуется минимизировать функцию методом карт Карно.

\end{document}