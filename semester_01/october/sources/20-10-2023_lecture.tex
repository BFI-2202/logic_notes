\documentclass{article}

\usepackage[T2A]{fontenc}
\usepackage[utf8]{inputenc}
\usepackage[russian]{babel}

\usepackage{tabularx}
\usepackage{amsmath}
\usepackage{pgfplots}
\usepackage{geometry}
\usepackage{multicol}
\geometry{
    left=1cm,right=1cm,top=2cm,bottom=2cm
}
\newcommand*\diff{\mathop{}\!\mathrm{d}}

\newtheorem{definition}{Определение}
\newtheorem{theorem}{Теорема}

\DeclareMathOperator{\sign}{sign}

\usepackage{hyperref}
\hypersetup{
    colorlinks, citecolor=black, filecolor=black, linkcolor=black, urlcolor=black
}

\title{Математическая логика и теория алгоритмов}
\author{Лисид Лаконский}
\date{October 2023}

\begin{document}
\raggedright

\maketitle

\tableofcontents
\pagebreak

\section{Лекция — 20.10.2023}

\subsection{Конструирование сложных высказываний}

Запишем базовые высказывания:

\begin{multicols}{2}
\begin{enumerate}
    \item Москва — столица России
    \item Саратов находится на берегу Невы
    \item Все люди смертны
    \item Сократ — человек
    \item Семь меньше четырех
    \item Волга впадает в Каспийское море
    \item Пушкин — великий русский математик
    \item Снег белый
\end{enumerate}
\end{multicols}

Из них мы можем сконструировать более сложные, например:

\begin{enumerate}
    \item Если Саратов находится на берегу Невы и все люди смертны, то Пушкин великий русский математик. Это выражение, которое можно записать как $(A_2 * A_3) \rightarrow A_7$, \textbf{истинно}.

    $(A_2 * A_3) \rightarrow A_7 = \lambda [(A_2 * A_3) \rightarrow A_7] = \lambda(A_2 * A_3) \rightarrow \lambda (A_7) = (\lambda (A_2) * \lambda(A_3)) \rightarrow \lambda(A_7) = (0 * 1) = 0 = 0 \rightarrow 0 = 1$
    \item Если Сократ человек и снег белый, то семь меньше четырех. Это выражение, которое можно записать как $(A_4 * A_8) \rightarrow A_5$, \textbf{ложно}.
\end{enumerate}

\subsection{Понятие формулы алгебры высказываний}

\begin{definition}
    Переменные, вместо которых можно подставлять высказывание (то есть, переменные, пробегающие множество высказываний), называют \textbf{пропозиционными переменными} (высказывательными переменными, переменными выскзываниями)
\end{definition}

В дальнейшем пропозиционные переменные будем обозначать заглавными латинскими буквами (или этими же буквами с индексами).

\begin{enumerate}
    \item Каждая отдельно взятая пропозиционная переменная является \textbf{формулой алгебры высказываний}
    \item Если $F_1$ и $F_2$ — формулы алгебры высказываний, то выражения, построенные путем их объединения с помощью логических операций, также будут являться \textbf{формулами алгебры высказываний}.
    \item Никаких других \textbf{формул алгебры высказываний} кроме получаемых с помощью пунктов 1 и 2 нет
\end{enumerate}

Подобные определения называются \textbf{индуктивными}. В них имеются

\begin{enumerate}
    \item \textbf{прямые пункты} (в данном случае 1 и 2), где задаются объекты, которые в дальнейшем именуются определяемым термином 
    \item \textbf{косвенный пункт} (в нашем сулчае 3), в котором говорится, что такие объекты исчерпываются объектами, заданными в прямых пунктах.
\end{enumerate}

Среди прямых пунктов имеются

\begin{enumerate}
    \item \textbf{базисные пункты} (в данном случае пункт 1), где указываются некоторые конкретные объекты, именуемые в дальнейшем определяемым термином;
    \item \textbf{индуктивные пункты} (в данном случае пункт 2), где даются правила получения определяемых объектов (в частности из объектов, перечисленных в базисных пунктах).
\end{enumerate}

\textbf{Индуктивный характер определения формулы позволяет решать две взаимнообратные задачи}:

\begin{enumerate}
    \item \textbf{Строить} новые все более сложные формулы из уже имеющихся;
    \item \textbf{Определять}, будет ли данное выражение, составленние из пропозиционных переменных, символов логических операций и скобок, формулой алгебры высказываний.
\end{enumerate}

Из определения следует, что для каждой формулы \textbf{существует конечная последовательность всех её подформул}. То есть, такая конечная последовательность, которая начинается с входящих в формулу пропозиционных переменных, заканчивается самой этой формулой и каждый член этой последовательности, не являющийся пропозиционной переменной, если бы отрицание уже имеющегося члена этой последовательности либо получается из двух уже имеющихся членов этой последовательности, их соединением с помощью одного из знаков логических операций и заключением полученного выражения в скобки.

Такую последовательность всех подформул данной формулы иногда называют \textbf{порождающей последовательностью} данной формулы и наличие такой последовательности у логического выражения \textbf{служит критерием того, что выражение является формулой}.

\end{document}