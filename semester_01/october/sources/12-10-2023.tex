\documentclass{article}

\usepackage[T2A]{fontenc}
\usepackage[utf8]{inputenc}
\usepackage[russian]{babel}

\usepackage{tabularx}
\usepackage{amsmath}
\usepackage{pgfplots}
\usepackage{geometry}
\usepackage{multicol}
\geometry{
    left=1cm,right=1cm,top=2cm,bottom=2cm
}
\newcommand*\diff{\mathop{}\!\mathrm{d}}

\newtheorem{definition}{Определение}
\newtheorem{theorem}{Теорема}

\DeclareMathOperator{\sign}{sign}

\usepackage{hyperref}
\hypersetup{
    colorlinks, citecolor=black, filecolor=black, linkcolor=black, urlcolor=black
}

\title{Математическая логика и теория алгоритмов}
\author{Лисид Лаконский}
\date{October 2023}

\begin{document}
\raggedright

\maketitle

\tableofcontents
\pagebreak

\section{Лекция — 12.10.2023}

\subsection{Алгебра высказываний}

Под \textbf{высказыванием} будем понимать будет понимать предложение, представляющее собой такое утверждение, о котором можно судить, истинно оно или ложно. По совокупности всех высказываний определяется \textbf{функция истинности}, принимающая значение ноль (если высказывание ложно) или один (если высказывание истинно):

$\lambda (P) = \begin{cases}
    1, \ \text{если высказывание } $P$ \text{ истинно} \\
    0, \ \text{если высказывание } $P$ \text{ ложно} \\
\end{cases}$

Функцию $\lambda(P)$ называют \textbf{логическим значением} (значением истинности) высказывания $P$.

\subsubsection{Логические операции}

Выражения связываются с помощью \textbf{логических операций}.

\textbf{Эквивалентностью} логических высказываний $P$ и $Q$ называется новое высказывание $P \leftrightarrow Q$ ($P$ эквивалентно $Q$; $P$ необходимо и достаточно для $Q$; $P$ тогда и только тогда, когда $Q$; $P$, если и только если $Q$), значение истинности которого задается следующей таблицей истинности:

$$
\begin{pmatrix}
    \lambda(P) & \lambda(Q) & \lambda(P \leftrightarrow Q) \\
    0 & 0 & 1 \\
    0 & 1 & 0 \\
    1 & 0 & 0 \\
    1 & 1 & 1
\end{pmatrix}
$$

\textbf{Импликацией} логических выражений $P$ и $Q$ называется новое высказывание $P \rightarrow Q$ (если $P$, то $Q$; из $P$ следует $Q$; $P$ влечёт $Q$; $P$ достаточно для $Q$; $Q$ необходимо для $P$), значение истинности которого задается следующей таблицей истинности:

$$
\begin{pmatrix}
    \lambda(P) & \lambda(Q) & \lambda(P \rightarrow Q) \\
    0 & 0 & 1 \\
    0 & 1 & 1 \\
    1 & 0 & 0 \\
    1 & 1 & 1
\end{pmatrix}
$$

В высказывании $P \rightarrow Q$ высказывание $P$ называется \textbf{посылкой}, а высказывание $Q$ называется \textbf{следствием}. Операцию импликации также называют \textbf{процессом рассуждения}.

\subsubsection{Операции над функциями истинности}

\begin{enumerate}
    \item \begin{enumerate}
        \item $0 \land 0 = 0$
        \item $0 \land 1 = 0$
        \item $\lambda(P \land Q) = \lambda(P) \land \lambda(Q)$
    \end{enumerate}
    \item \begin{enumerate}
        \item $\lnot 0 = 1$
        \item $\lnot 1 = 0$
        \item $\lambda(\lnot P) = \lnot \lambda (P)$
    \end{enumerate}
    \item $\lambda (P \lor Q) = \lambda (P) \lor \lambda (Q)$
    \item $\lambda (P \rightarrow Q) = \lambda (P) \rightarrow \lambda (Q)$
    \item $\lambda (P \leftrightarrow Q) = \lambda (P) \leftrightarrow \lambda (Q)$
\end{enumerate}

\end{document}